%%%% Proceedings format for most of ACM conferences (with the exceptions listed below) and all ICPS volumes.
\documentclass[sigconf]{acmart}
\usepackage{paralist}
\usepackage{url}
\usepackage[hyphenbreaks]{breakurl}

\def\UrlBreaks{\do\/\do-}

%%%% As of March 2017, [siggraph] is no longer used. Please use sigconf (above) for SIGGRAPH conferences.

%%%% Proceedings format for SIGPLAN conferences 
% \documentclass[sigplan, anonymous, review]{acmart}

%%%% Proceedings format for SIGCHI conferences
% \documentclass[sigchi, review]{acmart}

\usepackage{booktabs} % For formal tables
\usepackage{fullpage}
\usepackage{amsfonts}
\usepackage{times,eepic,epic,amssymb,amsfonts,
            fancybox,fullpage,color,rotating}
\usepackage{pstricks}
\usepackage{pst-node,pst-tree,pst-plot,pst-grad}

\usepackage[all,frame,line,color,dvips]{xy}

\parskip=1ex
\parindent=0em

\newcommand{\p}[1]{#1}
\newcommand{\bm}[1]{$#1$}
\newcommand{\wiggly}{\mbox{$\wr\!\!\wr\!\!\wr\;$}}

\newcommand{\centerbox}[1]{\begin{center}\ovalbox{#1}\end{center}}

\newcommand{\gt}[1]{\stackrel{#1}{\mbox{\large$\rightarrow$}}}
\newcommand{\gta}{\gt{a}}

\newcommand{\bfit}[1]{\textbf{\textit{#1}}}
\newcommand{\OFF}{\textsc{ Off}}
\newcommand{\ON}{\textsc{ On}}
\newcommand{\BROKEN}{\textsc{ Broken}}
\newcommand{\pull}{\textit{ pull}}
\newcommand{\bust}{\textit{ break}}
\newcommand{\reset}{\textit{ reset}}

\newcommand{\gameboard}{
\put(50,50){\makebox(0,0)[tr]{\bm a}}
\put(50,1550){\makebox(0,0)[br]{\bm a}}
\put(1350,50){\makebox(0,0)[tl]{\bm a}}
\put(1350,1550){\makebox(0,0)[bl]{\bm a}}
\put(50,800){\makebox(0,0){\bm a}}
\put(1350,800){\makebox(0,0){\bm a}}
\put(450,1250){\makebox(0,0)[bl]{\bm a}}
\put(950,1250){\makebox(0,0)[br]{\bm a}}
\put(450,400){\makebox(0,0)[br]{\bm b}}
\put(950,400){\makebox(0,0)[bl]{\bm a}}
\put(450,650){\makebox(0,0)[br]{\bm a}}
\put(950,650){\makebox(0,0)[bl]{\bm a}}
\put(720,780){\makebox(0,0)[l]{\bm a}}
\put(700,1250){\makebox(0,0)[b]{\bm a}}
\put(200,1400){\makebox(0,0){\large\bf 1}}
\put(1200,1400){\makebox(0,0){\large\bf 2}}
\put(700,1050){\makebox(0,0){\large\bf 3}}
\put(700,550){\makebox(0,0){\large\bf 4}}
\put(200,200){\makebox(0,0){\large\bf 5}}
\put(1200,200){\makebox(0,0){\large\bf 6}}
\qbezier(662,1142)(640,1171)(650,1200)
\qbezier(738,1142)(760,1171)(750,1200)
\qbezier(650,1200)(700,1270)(750,1200)
\path(733,1182)(738,1142)(770,1166)
\qbezier(108,162)(40,160)(40,110)
\qbezier(162,108)(160,40)(110,40)
\qbezier(40,110)(40,40)(110,40)
\path(177,73)(162,108)(137,73)
\qbezier(1238,108)(1240,40)(1290,40)
\qbezier(1292,162)(1360,160)(1360,110)
\qbezier(1290,40)(1360,40)(1360,110)
\path(1327,177)(1292,162)(1327,137)
\qbezier(1292,1438)(1360,1440)(1360,1490)
\qbezier(1238,1492)(1240,1560)(1290,1560)
\qbezier(1360,1490)(1360,1560)(1290,1560)
\path(1223,1527)(1238,1492)(1263,1527)
\qbezier(162,1492)(160,1560)(110,1560)
\qbezier(108,1438)(40,1440)(40,1490)
\qbezier(110,1560)(40,1560)(40,1490)
\path(73,1423)(108,1438)(73,1463)
\put(200,200){\circle{200}}
\put(200,1400){\circle{200}}
\put(700,550){\circle{200}}
\put(700,1050){\circle{200}}
\put(1200,200){\circle{200}}
\put(1200,1400){\circle{200}}
\path(282,1343)(618,1107) \path(568,1111)(618,1107)(597,1152)
\path(1118,1343)(782,1107) \path(803,1152)(782,1107)(832,1111)
\path(618,493)(282,257) \path(303,302)(282,257)(332,261)
\path(1118,257)(782,493) \path(832,489)(782,493)(803,448)
\path(700,950)(700,650) \path(675,693)(700,650)(725,693)
\path(251,286)(649,964) \path(649,914)(649,964)(606,939)
\path(1149,286)(751,964) \path(794,939)(751,964)(751,914)
\qbezier(162,1308)(0,800)(162,292) \path(122,322)(162,292)(168,342)
\qbezier(1238,1308)(1400,800)(1238,292) \path(1232,342)(1238,292)(1278,322)
}

% Copyright
%\setcopyright{none}
%\setcopyright{acmcopyright}
\setcopyright{acmlicensed}
%\setcopyright{rightsretained}
%\setcopyright{usgov}
%\setcopyright{usgovmixed}
%\setcopyright{cagov}
%\setcopyright{cagovmixed}

\copyrightyear{2019}
\acmYear{2019}
\setcopyright{acmlicensed}
\acmConference[ITiCSE '19]{The 24th Annual Conference on Innovation
  and Technology in Computer Science Education}{Jul. 15-17,
  2019}{Aberdeen, UK}
%\acmBooktitle{}
%\acmPrice{15.00}
%\acmDOI{10.1145/3159450.3159547}
%\acmISBN{978-1-4503-5103-4/18/02}
% This slight change to the code may also save 1 or 2 lines of space.

% removes the headers from each page per the preparation instructions, as these are not needed and will be updated with the chairs' actual session names during the pagination/indexing process:
\fancyhead{}

\begin{document}
\title{Formal Methods for First Years}
%\titlenote{}
%\subtitle{Extended Abstract}
%\subtitlenote{}
\author{Faron Moller}
\orcid{0000-0001-9535-8053}
\affiliation{%
  \institution{Swansea University}
  \streetaddress{}
  \city{Swansea} 
  \country{United Kingdom}
}
\email{f.g.moller@swansea.ac.uk}

\author{Tom Crick}
\orcid{0000-0001-5196-9389}
\affiliation{%
  \institution{Swansea University}
  \streetaddress{}
  \city{Swansea} 
  \country{United Kingdom}
}
\email{thomas.crick@swansea.ac.uk}

 
% The default list of authors is too long for headers}
%\renewcommand{\shortauthors}{Moller and Crick}


\begin{abstract}
In this paper, we describe the underlying concept and contents of our
new textbook \textit{``Modelling Computing Systems: The Mathematics of
Computer Science.''}  This book will be published by Springer in
Autumn 2013, and is aimed at first-year university computer science
students as a novel approach to introducing them in an engaging way to
formal methods at the very start of their education.  In fact, the
approach to formal modelling based on labelled transition systems
promoted by the book has been successfully adapted to workshops
delivered by Technocamps, a schools outreach programme aimed at
secondary school pupils.
\end{abstract}


\keywords{TBC}

\maketitle


\section{Why a New Book on Formal Methods}

Computer Science is a relatively young discipline.  University
Computer Science Departments are rarely more than a few decades
old. They will typically have emerged either from a Mathematics
Department or an Engineering Department, and until recently a Computer
Science degree was predominantly about writing computer programs (the
mathematical software) and building computers (the engineering
hardware).  Textbooks typically referred to programming as an ``art''
or a ``craft'' with little scientific basis compared to traditional
engineering subjects, and many computer programmers still like to see
themselves as part of a pop culture of geeks and hackers rather than
as academically-trained professionals.

However, the nature of Computer Science is changing rapidly,
reflecting the increasing ubiquity and importance of its subject
matter.  In the last decades, computational methods and tools have
revolutionised the sciences, engineering and technology. Computational
concepts and techniques are starting to influence the way we think,
reason and tackle problems; and computing systems have become an
integral part of our professional, economic and social lives.  The
more we depend on these systems -- particularly for safety-critical or
economically-critical applications -- the more we must ensure that
they are safe, reliable and well designed, and the less forgiving we
can be of failures, delays or inconveniences caused by the notorious
``computer glitch.''

Unlike traditional engineering disciplines which are solidly rooted on
centuries-old mathematical theories, the mathematical foundations
underlying Computer Science are younger, and Computer Scientists have
yet to agree on how best to approach the fundamental concepts and
tasks in the design of computing systems.  The Civil Engineer knows
exactly how to define and analyse a mathematical model of the
components of a bridge design so that it can be relied on not to fall
down, and the Aeronautical Engineer knows exactly how to define and
analyse a mathematical model of an aeroplane wing for the same
purpose.  However, Software Engineers have few universally-accepted
mathematical modelling tools at their disposal.  In the words of the
eminent Computer Scientist Alan Kay, ``most undergraduate degrees in
computer science these days are basically Java vocational training.''
But computing systems can be at least as complex as bridges or
aeroplanes, and a canon of mathematical methods for modelling
computing systems is therefore very much needed.  ``Software's Chronic
Crisis'' was the title of a popular and widely-cited Scientific
American article from 1994, and, unfortunately, its message remains
valid two decades later. 

University Computer Science Departments face a sociological challenge
posed by the fact that computers have become everyday, deceptively
easy-to-use objects.  A single generation ago, new Computer Science
students typically had teenage backgrounds spent writing Basic and/or
Assembly Language programs for their early hobbyist computers.  A
passion for this activity is what drove these students into University
Computer Science programmes, and they were not disappointed with the
education they received.  Their modern-day successors on the other
hand -- born directly into the heart of the computer era -- have grown
up with the internet, a billion dollar computer games industry, and
mobile phones with more computing power than the space shuttle.  They
often choose to study Computer Science on the basis of having a
passion for using computing devices throughout their everyday lives,
for everything from socialising with their friends to downloading the
latest films, and they often have less regard than they might to the
considerations of what a University Computer Science programme
entails, that it is far more than just using computers.

There is a universal trend of large numbers of first-year students
transferring out of Computer Science programmes and into related
programmes such as Media Studies or Information Studies.  This trend,
we feel, is often unjustified, and can be reversed by a more
considered approach to modelling and the mathematical foundations of
system design, one which the students can connect and feel at home
with right from the beginning of their University education.  This has
motivated us to produce a modern textbook, to be published by Springer
in Autumn 2013, aimed at teaching first-year undergraduate students
the essential mathematics and modelling techniques for computing
systems in a novel and relatively light-weight way.

The book is divided into two parts.  Part~I, subtitled
\emph{Mathematics for Computer Science}, introduces concepts from
Discrete Mathematics which are in the curriculum of any University
Computer Science programme, as well as much which often is not.  This
material is typically taught in service modules by mathematicians, and
new Computer Science students often find it difficult to engage with
the material presented in a purely mathematical context.  With this
book, we do our best to present the material in an engaging and
motivating fashion as the basis of computational thinking.

Part~II of the book, subtitled \emph{Modelling Computing Systems},
develops a particular approach to modelling based on state transition
systems.  Such transition systems have always featured in the Computer
Science curriculum, but traditionally (and increasingly historically)
only within the study of formal languages.  Here we introduce them as
general modelling devices, and explore languages and techniques for
expressing and reasoning about system specifications and (concurrent)
implementations.  Although Part~I covers twice as many pages as
Part~II, much of the Mathematics presented in Part~I itself is used
directly for modelling systems, and forms the basis on which the
approach developed in Part~II is based.

The main benefit of mathematical formalisation is that systems can be
modelled and analysed in precise and unambiguous ways; but formal
precision can also be a major pitfall in modelling since it can
compromise simplicity and intuition. In this book, therefore, we
always try to start from intuition and examples, and we aim at
developing precise concepts from that basis. How and when to be
precise is certainly not less important to learn than precision
itself: the ability to give mathematical proofs often does not depend
on knowing precise formal definitions and foundations.  One can, for
example, write down recursive functions without having a precise
formal concept in mind.

There is a long standing tradition in disciplines like Physics to
teach modelling through little artifacts. The fundamental ideas of
computational modelling and thinking as well can better be learned
from idealised examples and exercises than from many real world
computer applications. This book builds on a large collection of
logical puzzles and mathematical games that require no prior knowledge
about computers and computing systems; these can be much more fun and
sometimes much more challenging than analysing a device driver or a
criminal record database. Also, computational modelling and thinking
is about much more than just computers!

In fact, games play a far more important role in the book: they
provide a novel approach to understanding computer software and
systems which is proving to be very successful both in theory and
practice.  When a computer runs a program, for example, it is in a
sense playing a game against the user who is providing the input to
the program. The program represents a strategy which the computer is
using in this game, and the computer wins the game if it correctly
computes the result. In this game, the user is the adversary of the
computer and is naturally trying to confound the computer, which
itself is attempting to defend its claim that it is computing
correctly, that is, that the program it is running is a winning
strategy.  (In Software Engineering, this game appears in the guise of
\emph{testing}.)  Similarly, the controller of a software system that
interacts with its environment plays a game against the environment:
the controller tries to maintain the system's correctness properties,
while the environment tries to confound them.

This view suggests an approach to addressing three basic problems in
the design of computing systems:

\begin{enumerate}
\item \bfit{Specification} refers to the problem of precisely
identifying the task to be solved, as well as what exactly constitutes
a solution. This problem corresponds to the problem of defining a
winning strategy.
\item \bfit{Implementation} or \bfit{Synthesis} refers to the problem
of devising a solution to the task which respects the specification.
This problem corresponds to the problem of implementing a winning
strategy.
\item \bfit{Verification} refers to the problem of demonstrating that
the devised solution does indeed respect the specification.  This
problem corresponds to the problem of proving that a given strategy is
in fact a winning strategy.
\end{enumerate}

This analogy between the fundamental concepts in
Software Engineering on the one hand, and games and strategies on the
other, provides a mode of computational thinking which comes naturally
to the human mind, and can be readily exploited to explain and
understand Software Engineering concepts and their applications.  It
also motivates our thesis that Game Theory provides a paradigm for
understanding the nature of computation. 

There are over 200 exercises presented throughout the book, all of
which have complete solutions at the back of the book; as well as over
200 further exercises at the ends of the chapters whose solutions are
not provided.  The exercises within the chapters are often used to
explore subtleties or side-issues, or simply to put lengthy arguments
into an appendix.  The material in this book has been used
successfully for over a decade in first-year Discrete Mathematics and
Systems Modelling modules. Countless eyes have passed over the text,
and a thousand students have solved its exercises.

\newcommand{\fb}[1]{\Big(\makebox[2.0em]{\tiny$\begin{array}{@{}c@{}c@{}c@{}}#1\end{array}$}\Big)}
\newcommand{\nofb}[1]{\makebox[2.0em]{\tiny$\begin{array}{@{}c@{}c@{}c@{}}#1\end{array}$}}

\section{Labelled Transition Systems for Problem Solving}

Consider the following presentation of Euclid's algorithm
for computing the greatest common divisor
of two numbers \texttt{x} and \texttt{y}:
\begin{quote}
\texttt{forever do}
\vspace{-2ex}
\begin{quote}
\texttt{x\,:$\!$=\,x\,\,mod\,\,y;} \\
\texttt{if x=0 then return y;} \\
\texttt{y\,:$\!$=\,y\,\,mod\,\,x;} \\
\texttt{if y=0 then return x}
\vspace{-2ex}
\end{quote}
\texttt{od}
\end{quote}
To understand this program,
you can hand-turn it, keeping track of
the state of the variables:
\begin{center}
\setlength{\unitlength}{0.005\textwidth}
\begin{picture}(110,40)(0,0)
\scriptsize
\put(10,5){\ellipse{16}{10}}
\put(10,5){\makebox(0,0){$\begin{array}{r@{=}l} x & 72
                          \\[-0.5ex] y & 174 \end{array}$}}
\put(10,35){\ellipse{16}{10}}
\put(10,35){\makebox(0,0){$\begin{array}{r@{=}l} x & 246
                          \\[-0.5ex] y & 174 \end{array}$}}
\put(50,5){\ellipse{14}{10}}
\put(50,5){\makebox(0,0){$\begin{array}{r@{=}l} x & 12
                          \\[-0.5ex] y & 30 \end{array}$}}
\put(50,35){\ellipse{14}{10}}
\put(50,35){\makebox(0,0){$\begin{array}{r@{=}l} x & 72
                          \\[-0.5ex] y & 30 \end{array}$}}
\put(90,5){\ellipse{14}{10}}
\put(90,5){\makebox(0,0){$\begin{array}{r@{=}l} x & 0
                          \\[-0.5ex] y & 6 \end{array}$}}
\put(90,35){\ellipse{14}{10}}
\put(90,35){\makebox(0,0){$\begin{array}{r@{=}l} x & 12
                          \\[-0.5ex] y & 6 \end{array}$}}
\put(11,25){\makebox(0,0)[l]{\texttt{x\,:$\!$=\,x\,\,mod\,\,y}}}
\put(51,25){\makebox(0,0)[l]{\texttt{x\,:$\!$=\,x\,\,mod\,\,y}}}
\put(91,25){\makebox(0,0)[l]{\texttt{x\,:$\!$=\,x\,\,mod\,\,y}}}
\path(10,30)(10,10) \path(9,12)(10,10)(11,12)
\path(50,30)(50,10) \path(49,12)(50,10)(51,12)
\path(90,30)(90,10) \path(89,12)(90,10)(91,12)
\path(14.8,9)(45.2,31.4) \path(43.0,31.0)(45.2,31.4)(44.2,29.6)
\path(54.8,8.6)(85.2,31.4) \path(83.0,31.0)(85.2,31.4)(84.2,29.6)
\put(26,16){\makebox(0,0)[l]{\texttt{y\,:$\!$=\,y\,\,mod\,\,x}}}
\put(66,16){\makebox(0,0)[l]{\texttt{y\,:$\!$=\,y\,\,mod\,\,x}}}
\end{picture}
\end{center}
In general, a {computation}
-- or more generally a {process} --
can be represented by a {Labelled Transition System} ({LTS}),
which consists of a {directed graph},
where the {vertices} represent {states},
and the edges represent {transitions}
from state to state, and are labelled by {events}.
As shown above, an LTS is typically presented pictorially,
with the {states} represented by circles
and the {transitions} by {arrows} between states
labelled by {actions}.

As a further example, consider the following lamp process:
\begin{center}
{
\setlength{\unitlength}{0.004\textwidth}
\begin{picture}(40,50)(0,-5)
\path(0,25)(10,40)(30,40)(40,25)(0,25)
\path(10,0)(10,8)(30,8)(30,0)(10,0)
\path(25,25)(25,15)
\put(25,15){\circle*{3}}
\put(27,3){\circle*{3}}
\Thicklines
\path(19.25,8)(19.25,25) \path(19.5,8)(19.5,25) \path(19.75,8)(19.75,25)
\path(20,8)(20,25)
\path(20.25,8)(20.25,25) \path(20.5,8)(20.5,25) \path(20.75,8)(20.75,25)
\end{picture}
}
\hspace{5em}
\setlength{\unitlength}{0.0020\textwidth}
\begin{picture}(160,100)(0,-10)
\scriptsize
\put(50,75){\oval(30,20)}
\put(150,75){\oval(30,20)}
\put(100,25){\oval(40,20)}
\put(50,75){\makebox(0,0){\OFF}}
\put(150,75){\makebox(0,0){\ON}}
\put(100,25){\makebox(0,0){\BROKEN}}
\path(65,80)(135,80) \path(130,82)(135,80)(130,78)
\path(135,70)(65,70) \path(70,72)(65,70)(70,68)
\path(50,65)(50,40) \path(80,25)(65,25) \path(48,60)(50,65)(52,60)
\put(65,40){\oval(30,30)[bl]}
\path(150,65)(150,40) \path(120,25)(135,25) \path(125,27)(120,25)(125,23)
\put(135,40){\oval(30,30)[br]}
\put(100,15){\oval(20,30)[b]} \path(108,10)(110,15)(112,10)
\put(100,-2){\makebox(0,0)[t]{\pull}}
\put(100,82){\makebox(0,0)[b]{\pull}}
\put(100,68){\makebox(0,0)[t]{\pull}}
\put(48,45){\makebox(0,0)[r]{\reset}}
\put(152,45){\makebox(0,0)[l]{\bust}}
\end{picture}
\end{center}
The lamp has a string to pull for turning the light on and off,
and a reset button which resets
the circuit if a built-in circuit breaker breaks
when the light is on.

At any moment in time the lamp can be in one of three states:

\begin{itemize}
\item
\OFF\ -- in which the light is off (and the circuit breaker is set);
\item
\ON\ -- in which the light is on (and the circuit breaker is set);
 \,and
\item
\BROKEN\ -- in which the circuit breaker is broken (and the light is off).
\end{itemize}

In any state the string can be pulled, causing a transition into the
appropriate new state (from the state \BROKEN, the new state is the
same state \BROKEN).  In the state \ON, the circuit breaker may break,
causing a transition into the state \BROKEN\ in which the reset button
has popped out; from this state, the reset button may be pushed,
causing a transition into the state \OFF.

These simple examples demonstrate the simple, but effective, use of
LTSs as a means of modelling computing problems and real world
objects.  Of course, LTSs are not limited to such primitive forms.
They can be extended in a variety of ways to add further information,
for example, notions of time and space can be represented within
states so that real-time and hybrid systems can be described.  In this
respect, LTSs can be regarded as a general formalism for modelling any
kind of system, be it a computing system, a real world object, or a
concurrent real-time system with multiple components.

\paragraph{Introducing LTSs with Puzzles}
\newblock

Whilst the definition of a labelled transition system is surprisingly
straightforward for such a powerful formalism, getting students to
engage with it requires some ingenuity.  Fortunately, this is equally
straightforward by resorting to well-known recreational puzzles.

\paragraph{The Man-Wolf-Goat-Cabbage Riddle}
\newblock

\begin{quote}
\emph{A man needs to cross a river with a wolf, a goat and a cabbage.
His boat is only large enough to carry himself and one of his three
possessions, so he must transport these items one at a time.  However,
if he leaves the wolf and the goat together unattended, then the wolf
will eat the goat; similarly, if he leaves the goat and the cabbage
together unattended, then the goat will eat the cabbage.  How can the
man get across safely with his three items?}
\end{quote}

This riddle was posed by Alcuin of York in the 8th century, and more
recently tackled by Homer Simpson in a 2009 episode of The Simpsons
titled Gone Maggie Gone.

This puzzle can be solved by modelling it as an LTS.  A state of the
LTS will represent the current position (left or right bank) of the
four entities (man, wolf, goat, cabbage); and there will be four
actions representing the four possible actions that the man can take:

\begin{itemize}
\item $m$ = the man crosses the river on his own;
\item $w$ = the man crosses the river with the wolf;
\item $g$ = the man crosses the river with the goat; and
\item $c$ = the man crosses the river with the cabbage.
\end{itemize}

\begin{center}
~
\begin{xy}
*=<10ex,3ex>[o]{},{*[grey]\frm{**}},
*=<10ex,3ex>[o][F-]{\textsc{wgc : m}}="WGC"
\ar@/^1ex/^{m}
   + <0ex,-5.5ex>*=<10ex,3ex>[o][F-]{\textsc{mwgc :}}="MWGC"
\ar@{<-}@/_1ex/_{m} "MWGC"

\POS "MWGC"
\ar@/^1ex/^{g}
   + <0ex,-5.5ex>*=<10ex,3ex>[o][F-]{\textsc{wc : mg}}="WC"
\ar@{<-}@/_1ex/_{g} "WC"
\ar@/_3ex/_{w}
   + <-20ex,-15ex>*=<10ex,3ex>[o]{},{*[grey]\frm{**}},
                  *=<10ex,3ex>[o][F-]{\textsc{gc : mw}}="GC"
\ar@{<-}@/^1ex/^{w} "GC"
\ar@/^3ex/^{c}
   + <20ex,-15ex>*=<10ex,3ex>[o]{},{*[grey]\frm{**}},
                 *=<10ex,3ex>[o][F-]{\textsc{wg : mc}}="WG"
\ar@{<-}@/_1ex/_{c} "WG"

\POS "WC"
\ar@/^1ex/^{m}
   + <0ex,-5.5ex>*=<10ex,3ex>[o][F-]{\textsc{mwc : g}}="MWC"
\ar@{<-}@/_1ex/_{m} "MWC"

\POS "MWC"
\ar@{<-}@/^1ex/^{w}
   + <-5.5ex,-5.5ex>*=<10ex,3ex>[o][F-]{\textsc{c : mwg}}="C"
\ar@/_1ex/_{w} "C"
\ar@/^1ex/^{c}
   + <6ex,-5.5ex>*=<10ex,3ex>[o][F-]{\textsc{w : mcg}}="W"
\ar@{<-}@/_1ex/_{c} "W"

\POS "C"
\ar@/^1ex/^{g}
   + <0ex,-5.5ex>*=<10ex,3ex>[o][F-]{\textsc{mgc : w}}="MGC"
\ar@{<-}@/_1ex/_{g} "MGC"

\POS "W"
\ar@/^1ex/^{g}
   + <0ex,-5.5ex>*=<10ex,3ex>[o][F-]{\textsc{mwg : c}}="MWG"
\ar@{<-}@/_1ex/_{g} "MWG"

\POS "MGC"
\ar@/^1ex/^{c}
   + <6ex,-5.5ex>*=<10ex,3ex>[o][F-]{\textsc{g : mwc}}="G"
\ar@{<-}@/_1ex/_{c} "G"

\POS "MWG"
\ar@{<-}@/^1ex/^{w} "G"
\ar@/_1ex/_{w} "G"

\POS "G"
\ar@/^1ex/^{m}
   + <0ex,-5.5ex>*=<10ex,3ex>[o][F-]{\textsc{mg : wc}}="MG"
\ar@{<-}@/_1ex/_{m} "MG"

\POS "MG"
\ar@/^1ex/^{g}
   + <0ex,-5.5ex>*=<10ex,3ex>[o][F-]{\textsc{ : mwgc}}=""
\ar@{<-}@/_1ex/_{g} ""

\POS ""
\ar@/^1ex/^{m}
   + <0ex,-5.5ex>*=<10ex,3ex>[o]{},{*[grey]\frm{**}},
               *=<10ex,3ex>[o][F-]{\textsc{m : wgc}}="M"
\ar@{<-}@/_1ex/_{m} "M"
\ar@/^3ex/^{c}
   + <-20ex,15ex>*=<10ex,3ex>[o]{},{*[grey]\frm{**}},
                 *=<10ex,3ex>[o][F-]{\textsc{mc : wg}}="MC"
\ar@{<-}@/_1ex/_{c} "MC"
\ar@/_3ex/_{w}
   + <20ex,15ex>*=<10ex,3ex>[o]{},{*[grey]\frm{**}},
                *=<10ex,3ex>[o][F-]{\textsc{mw : gc}}="MW"
\ar@{<-}@/^1ex/^{w} "MW"

\POS "WG"
\ar@/^2ex/^{m} "MWG"
\ar@{<-}@/_1ex/_{m} "MWG"

\POS "GC"
\ar@/_2ex/_{m} "MGC"
\ar@{<-}@/^1ex/^{m} "MGC"

\POS "MC"
\ar@/^2ex/^{m} "C"
\ar@{<-}@/_1ex/_{m} "C"

\POS "MW"
\ar@{<-}@/^1ex/^{m} "W"
\ar@/_2ex/_{m} "W"
\end{xy}
~
\end{center}

The initial state is \ovalbox{MWGC :} (meaning all are on the left
bank of the river).

We wish to find a sequence of actions which will lead to the state
\ovalbox{: MWGC} (meaning all are on the right bank of the river).

However, we want to avoid going through any of the six dangerous
states:

\begin{center}
\begin{tabular}{ccc}
\ovalbox{WGC : M} & \ovalbox{GC : MW} & \ovalbox{WG : MC}
\\[1ex]
\ovalbox{MC : WG} & \ovalbox{MW : GC} & \ovalbox{M : WGC}
\end{tabular}
\end{center}
There are several possibilities (all involving at least 7 crossings),
for example
\begin{center}
$g$, $m$, $w$, $g$, $c$, $m$, $g$.
\end{center}

\paragraph{The Missionaries and Cannibals Riddle}
\newblock
\begin{quote}
\emph{Three missionaries are travelling with three cannibals when they
come upon a river.  They have a boat, but it can only hold two people.
The river is filled with piranha, so they all must eventually cross in
the boat; no one can cross the river by swimming.  The problem is:
should the cannibals ever outnumber the missionaries on either side of
the river, the outnumbered missionaries would be in deep trouble.
Each missionary and each cannibal can row the boat.  How can all six
get across the river safely?  }
\end{quote}

Similarly to the Man-Wolf-Goat-Cabbage riddle, this puzzle can also be
solved using an LTS, as depicted in Figure~\ref{fig:MC}.
\begin{figure}[t]
\centering
 ~ 
\begin{xy}
 *=<16ex,8ex>[o][F-]{\fb{m&m&m\\[-1ex]c&c&c} \;\wiggly \nofb{\cdot\\[-1ex]\cdot}}="33L"
 + <0ex,-9ex>
 *=<16ex,8ex>[o][F-]{\fb{m&m&m\\[-1ex]c&c} \;\wiggly \nofb{\cdot\\[-1ex]c}}="32L"
 + <0ex,-9ex>
 *=<16ex,8ex>[o][F-]{\fb{m&m&m\\[-1ex]c} \;\wiggly \nofb{\cdot\\[-1ex]c&c}}="31L"
 + <0ex,-9ex>
 *=<16ex,8ex>[o][F-]{\fb{m&m\\[-1ex]c&c} \;\wiggly \nofb{m\\[-1ex]c}}="22L"
 + <0ex,-9ex>
 *=<16ex,8ex>[o][F-]{\fb{m\\[-1ex]c} \;\wiggly \nofb{m&m\\[-1ex]c&c}}="11L"
 + <0ex,-9ex>
 *=<16ex,8ex>[o][F-]{\fb{\cdot\\[-1ex]c&c&c} \;\wiggly \nofb{m&m&m\\[-1ex]\cdot}}="03L"
 + <0ex,-9ex>
 *=<16ex,8ex>[o][F-]{\fb{\cdot\\[-1ex]c&c} \;\wiggly \nofb{m&m&m\\[-1ex]c}}="02L"
 + <0ex,-9ex>
 *=<16ex,8ex>[o][F-]{\fb{\cdot\\[-1ex]c} \;\wiggly \nofb{m&m&m\\[-1ex]c&c}}="01L"
\POS "33L"
 + <45ex,4ex>
 *=<16ex,8ex>[o][F-]{\nofb{m&m&m\\[-1ex]c&c} \;\wiggly \fb{\cdot\\[-1ex]c}}="32R"
 + <0ex,-9ex>
 *=<16ex,8ex>[o][F-]{\nofb{m&m&m\\[-1ex]c} \;\wiggly \fb{\cdot\\[-1ex]c&c}}="31R"
 + <0ex,-9ex>
 *=<16ex,8ex>[o][F-]{\nofb{m&m&m\\[-1ex]\cdot} \;\wiggly \fb{\cdot\\[-1ex]c&c&c}}="30R"
 + <0ex,-9ex>
 *=<16ex,8ex>[o][F-]{\nofb{m&m\\[-1ex]c&c} \;\wiggly \fb{m\\[-1ex]c}}="22R"
 + <0ex,-9ex>
 *=<16ex,8ex>[o][F-]{\nofb{m\\[-1ex]c} \;\wiggly \fb{m&m\\[-1ex]c&c}}="11R"
 + <0ex,-9ex>
 *=<16ex,8ex>[o][F-]{\nofb{\cdot\\[-1ex]c&c} \;\wiggly \fb{m&m&m\\[-1ex]c}}="02R"
 + <0ex,-9ex>
 *=<16ex,8ex>[o][F-]{\nofb{\cdot\\[-1ex]c} \;\wiggly \fb{m&m&m\\[-1ex]c&c}}="01R"
 + <0ex,-9ex>
 *=<16ex,8ex>[o][F-]{\nofb{\cdot\\[-1ex]\cdot} \;\wiggly \fb{m&m&m\\[-1ex]c&c&c}}="00R"

\POS "33L"
  \ar@{<->}@/^5ex/^{c} "32R"
  \ar@{<->}@/^5ex/^{cc} "31R"
  \ar@{<->}@/^2ex/^{mc} "22R"
\POS "32L"
  \ar@{<->}@/^5ex/^{c} "31R"
  \ar@{<->}@/_1ex/^{cc} "30R"
  \ar@{<->}@/_3ex/^{m} "22R"
\POS "31L"
  \ar@{<->}@/^1ex/^{c} "30R"
  \ar@{<->}@/^2ex/^{mm} "11R"
\POS "22L"
  \ar@{<->}@/^1ex/^{mc} "11R"
  \ar@{<->}@/^1ex/^{mm} "02R"
\POS "11L"
  \ar@{<->}@/^5ex/^{m} "01R"
  \ar@{<->}@/_1ex/^{mc} "00R"
\POS "03L"
  \ar@{<->}@/^1ex/^{c} "02R"
  \ar@{<->}@/_5ex/^{cc} "01R"
\POS "02L"
  \ar@{<->}@/_4ex/^{c} "01R"
  \ar@{<->}@/_3ex/^{cc} "00R"
\POS "01L"
  \ar@{<->}@/_4ex/^{c} "00R"
\end{xy}
 ~ 
\caption{The Missionaries and Cannibals riddle as an LTS.}
\label{fig:MC}
\end{figure}

Each state of the LTS records the positions of the people (which banks
they are on) and which side holds the boat.  The groups on the two
banks are depicted side-by-side divided by wiggly lines representing
the river, with the group holding the boat enclosed in parentheses. We
only consider the safe states where the cannibals do not outnumber the
missionaries.

There are five possible actions:

\begin{itemize}
\item $m$ (a missionary crosses alone);
\item $mm$ (two missionaries cross together);
\item $c$ (a cannibal crosses alone);
\item $cc$ (two cannibals cross together); and
\item $mc$ (a missionary and a cannibal cross together).
\end{itemize}

Notice that all of the transitions are drawn bi-directionally, as
every transition can clearly be reversed.

The group start in the top-left state in which the whole group is on
the left bank, and they wish to get to the bottom-right state in which
they are all on the right bank.  It is not hard to find a such path
through the LTS which involves $11$ crossings.

\paragraph{The Water Jugs Riddle}
\newblock

In the 1995 film {Die Hard: With a Vengeance}, New York Detective John
McClane (played by Bruce Willis) and Harlem dry cleaner Zeus Carver
(played by Samuel L. Jackson) had to solve the following problem in
order to prevent a bomb from exploding at a public fountain.

\begin{quote}
\emph{Given only a five-gallon jug and a three-gallon jug, neither
with any markings on them, they had to fill the larger jug with
\emph{exactly} four gallons of water from the fountain, and place it
onto a scale in order to stop the bomb's timer and prevent disaster.
}
\end{quote}

How did they manage this feat?

\vspace{3ex}

This riddle was posed by Abbot Albert in the 13th Century.

A state of the system underlying this riddle consists of a pair of
integers $(i,j)$ with $0{\leq}i{\leq}5$ and $0{\leq}j{\leq}3$,
representing the volume of water in the 5-gallon and 3-gallon jugs $A$
and $B$, respectively.  The initial state is $(0,0)$ and the final
state you wish to reach is $(4,0)$.

There are six moves possible from a given state $(i,j)$:
\newcommand{\lgt}[1]{\stackrel{#1}{\mbox{\large $\longrightarrow$}}}

\begin{center}
\small
$\begin{array}{llll}
$(i,j)$ & \lgt{\textit{\small fillA}}
        & (5,j) & \bigg(\textit{if } i{=}0\bigg) \\[1ex]
$(i,j)$ & \lgt{\textit{\small fillB}}
        & (i,3) & \bigg(\textit{if } j{=}0\bigg) \\[1ex]
$(i,j)$ & \lgt{\textit{\small emptyA}}
        & (0,j) & \bigg(\textit{if } i{>}0\bigg) \\[1ex]
$(i,j)$ & \lgt{\textit{\small emptyB}}
        & (i,0) & \bigg(\textit{if } j{>}0\bigg) \\[1ex]
$(i,j)$ & \lgt{\textit{\small AtoB}}
        & \bigg(\max(0,i{+}j{-}3),\min(3,i{+}j)\bigg)
        & \bigg(\textit{if } i{>}0 \textit{ and } j{<}3\bigg) \\[1ex]
$(i,j)$ & \lgt{\textit{\small BtoA}}
        & \bigg(\min(5,i{+}j),\max(0,i{+}j{-}5)\bigg)
        & \bigg(\textit{if } i{<}5 \textit{ and } j{>}0\bigg) \\[1ex]
\end{array}$
\end{center}
Drawing out the LTS, we get the following $7$-step solution:
\begin{quote}
$(0,0) \lgt{\textit{\small fillA}} (5,0) \lgt{\textit{\small AtoB}} (2,3)
       \lgt{\textit{\small emptyB}} (2,0) \lgt{\textit{\small AtoB}} (0,2)$

\hfill
$\lgt{\textit{\small fillA}} (5,2) \lgt{\textit{\small AtoB}} (4,3)
       \lgt{\textit{\small emptyB}} (4,0)$.
\end{quote}

These simple riddles and puzzles allow students to easily grasp and
understand the powerful concept of labelled transition systems. After
seeing only a few examples, they are able to model straightforward
systems by themselves using LTSs. Once an intuitive understanding has
been established, the task of understanding the mathematics behind
LTSs becomes less foreboding.

\section{Bisimulation for Dummies}

Beyond having a formalism for representing and simulating (the
behaviour of) a system, we want to be able to determine if the system
is correct.  In its most basic form, this amounts to determining if
the system matches its specification, where we assume that both the
system and its specification are given as states of some LTS.  For
example, consider the two vending machines $V_1$ and $V_2$ depicted
below, where $V_1$ is taken to represent the specification of the
vending machine while $V_2$ is taken to represent its implementation.

\begin{center}
\parbox{0.3\textwidth}{
\setlength{\unitlength}{0.0004\textwidth}
\begin{picture}(500,650)(0,0)
\Thicklines

\path(0,0)(0,600)(500,600)(500,0)(0,0)

\thinlines

\put(100,490){\circle*{25}}
\put(400,490){\circle*{25}}
\put(100,490){\circle{60}}
\put(400,490){\circle{60}}

\path(230,325)(230,450)(270,450)(270,325)(230,325)
\path(100,100)(100,250)(400,250)(400,100)(100,100)
\path(125,125)(125,225)(375,225)(375,125)(125,125)
\path(100,100)(125,125) \path(100,250)(125,225)
\path(400,250)(375,225) \path(400,100)(375,125)

\put(100,550){\makebox(0,0){\tiny \p{coffee}}}
\put(400,550){\makebox(0,0){\tiny \p{tea}}}
\put(250,290){\makebox(0,0){\tiny \p{10p}}}
\put(250,70){\makebox(0,0){\tiny \p{collect}}}

\end{picture}
}
\parbox{0.45\textwidth}{
\setlength{\unitlength}{0.00040\textwidth}
\p{\footnotesize
\psset{unit=0.065\textwidth,arrowsize=0.20,arcangle=15,nrot=:U,labelsep=0.05}
\begin{pspicture}(-0.5,-0.5)(6.4,1.6)%\psgrid
\cnode(0.0,0.5){0.3}{1}
\cnode(2.0,1.0){0.3}{2}
\cnode(4.0,1.0){0.3}{3}
\cnode(6.0,0.5){0.3}{4}
\rput(0.0,0.5){$V_1$}
\ncarc{->}{1}{2}\naput{$\p{10p}$}
\ncarc{->}{2}{3}\naput{$\p{10p}$}
\ncarc{->}{3}{4}\naput{$\p{coffee}$}
\ncarc[arcangle=-10]{->}{3}{4}\nbput{$\p{tea}$}
\ncarc[arcangle=-20]{<-}{1}{4}\naput{$\p{collect}$}
\end{pspicture}
}
\setlength{\unitlength}{0.00040\textwidth}
\p{\footnotesize
\psset{unit=0.065\textwidth,arrowsize=0.20,arcangle=15,nrot=:U,labelsep=0.05}
\begin{pspicture}(-0.5,-0.8)(6.4,1.4)%\psgrid
\cnode(0.0,0.5){0.3}{1}
\cnode(2.0,0.5){0.3}{2}
\cnode(4.0,0.9){0.3}{3C}
\cnode(4.0,0.1){0.3}{3T}
\cnode(6.0,0.5){0.3}{4}
\rput(0.0,0.5){$V_2$}
\ncarc{->}{1}{2}\naput{$\p{10p}$}
\ncarc{->}{2}{3C}\naput{$\p{10p}$}
\ncarc[arcangle=-15]{->}{2}{3T}\naput{$\p{10p}$}
\ncarc{->}{3C}{4}\naput{$\p{coffee}$}
\ncarc[arcangle=-15]{->}{3T}{4}\nbput{$\p{tea}$}
\ncarc[arcangle=-60,ncurv=0.5]{<-}{1}{4}\naput{$\p{collect}$}
%\nccurve[angleA=250,angleB=-45,ncurv=0.3]{->}{4}{1}\nbput{$\p{collect}$}
\end{pspicture}
}}
\end{center}

Clearly the behaviour of \bm{V_1} is somehow different from the
behaviour of \bm{V_2}: after \textit{twice} inserting a \p{10p} coin
into \bm{V_1}, we are \textit{guaranteed} to be \textit{able} to press
the \p{coffee} button; this is \textit{not} true of \bm{V_2}.  The
question is: \emph{How do we formally distinguish between processes?}

\paragraph{The formal definition of bisimilarity}
\newblock

A traditional approach to this question relies on
determining if these two states are related by
a \emph{bisimulation relation} $R$ as defined as follows.
\begin{quote}
\emph{A binary relation $R$ over states of an LTS
is a bisimulation relation if, and only if,
whenever $(x,y) \in R$:
\begin{itemize}
\item
if $x \gta x'$ for some $x'$ and $a$,
then $y \gta y'$ for some $y'$ such that $(x',y') \in R$;
and
\item
if $y \gta y'$ for some $y'$ and $a$,
then $x \gta x'$ for some $x'$ such that $(x',y') \in R$.
\end{itemize}
}
\end{quote}

Simple inductive definitions already represent a major challenge for
undergraduate university students; so it is no surprise that this
coinductive definition of a bisimulation relation is incomprehensible
even to some of the brightest postgraduate students -- at least on
their first encounter with it.  However, there is a straightforward
way to explain the idea of bisimilarity to first-year students -- a
way which they can readily grasp and are happy to explore and, indeed,
play with.  The approach is based on the following game.

\paragraph{The Copy-Cat Game}
\newblock

This game is played between two players, typically referred to as
Alice and Bob.  We start by placing tokens on two states of an LTS,
and then proceed as follows.

\begin{enumerate}
\item
The {first} player (Alice) chooses one of the two tokens,
and moves it forward along an arrow to another state;
if this is impossible (that is, if there are no arrows
leading out of either node on which the tokens sit), then
the second player (Bob) is declared to be the winner.
\item The {second} player (Bob) must move the \emph{other} token
  forward along an arrow which has \textit{the same label} as the
  arrow used by the first player; if this is impossible, then the
  first player (Alice) is declared to be the winner.
\end{enumerate}

This exchange of moves is repeated for as long as neither player gets
stuck.  If the second player ever gets stuck, then the first player is
declared to be the winner; otherwise the second player is declared to
be the winner (in particular, if the game goes on forever).

Alice, therefore, wants to show that the two states holding tokens are
somehow different, in that there is something that can happen from one
of the two states which cannot happen from the other.  Bob, on the
other hand, wants to show that the two states are the same: that
whatever might happen from one of the two states can be copied by the
other state.

It is easy to argue that two states should be considered equivalent
exactly when Bob has a winning strategy in the Copy-Cat Game starting
with the tokens on the two states in question; and indeed this is
taken to be the definition of when two states are equal, specifically,
when an implementation matches its specification.

%%%%%%%%%%%
%%%%%%%%%%%
%\newpage
%%%%%%%%%%%
%%%%%%%%%%%

As an example, consider playing the game on the following LTS.

\begin{center}
\psset{unit=0.08\textwidth,arrowsize=0.15,arcangle=15,nrot=:U,labelsep=0.05}
\begin{pspicture}(-2.3,-1.0)(6.3,1.0)%\psgrid
\cnode(-2,0){0.3}{U}
\rput(-2,0){\large U}
\cnode(2,0){0.3}{Z}
\rput(2,0){\large Z}
\cnode(4,0){0.3}{Y}
\rput(4,0){\large Y}
\cnode(6,0){0.3}{X}
\rput(6,0){\large X}
\cnode(0,0.8){0.3}{V}
\rput(0,0.8){\large V}
\cnode(0,-0.8){0.3}{W}
\rput(0,-0.8){\large W}
\ncarc[arcangle=-10]{->}{U}{W}\naput{\large $a$}
\ncarc[arcangle=10]{->}{U}{V}\naput{\large $a$}
\ncarc[arcangle=-10]{->}{W}{Z}\naput{\large $c$}
\ncarc[arcangle=10]{->}{V}{Z}\naput{\large $b$}
\ncarc[arcangle=35]{<-}{Z}{Y}\naput{\large $c$}
\ncarc[arcangle=-35]{<-}{Z}{Y}\nbput{\large $b$}
\ncarc[arcangle=0]{<-}{Y}{X}\naput{\large $a$}
\end{pspicture}
\end{center}
Starting with tokens on states $U$ and $X$,
the \emph{first} player (Alice) has a winning strategy:
\begin{itemize}
\item
Alice can move the token on $U$
along the $a$-transition to $V$.
\item
Bob must match this by
moving the token on $X$ along the $a$-transition to $Y$.
\item
Alice can then move the token on $Y$
along the $c$-transition to $Z$.
\item
Bob will be stuck, as there is no $c$-transition from $V$.
\end{itemize}

This example is a simplified version of the vending machine example;
and a straightforward adaptation of the winning strategy for Alice
will work in the game starting with the tokens on the states $V_1$ and
$V_2$.  We thus have an argument as to why the two vending machines
are different.

\paragraph{Relating winning strategies to bisimilarity}
\newblock

Whilst this notion of equality between states is particularly simple
and even entertaining to explore, it coincides precisely with the
complicated coinductive definition of when two states are bisimilar.
Furthermore, seeing this is the case is almost equally
straightforward.

\begin{itemize}
\item
Suppose we play the Copy-Cat Game starting with the tokens on two
states $E$ and $F$ which are related by some bisimulation relation
$R$.  It is easy to see that Bob has a winning strategy: whatever move
Alice makes, by the definition of a bisimulation relation, Bob will be
able to copy this move in such a way that the two tokens will end up
on states $E'$ and $F'$ which are again related by $R$; and Bob can
keep repeating this for as long as the game lasts, meaning that he
wins the game.
\item
Suppose now that $R$ is the set of pairs of states of an LTS from
which Bob has a winning strategy in the Copy-Cat Game.  It is easy to
see that this is a bisimulation relation: suppose that $(x,y) \in R$:
\begin{itemize}
\item
if $x \gta x'$ for some $x'$ and $a$, then taking this to be a move by
Alice in the Copy-Cat Game, we let $y \gta y'$ be a response by Bob
using his winning strategy; this would mean that Bob still has a
winning strategy from the resulting pair of states, that is $(x',y')
\in R$;
\item
if $y \gta y'$ for some $y'$ and $a$, then taking this to be a move by
Alice in the Copy-Cat Game, we let $x \gta x'$ be a response by Bob
using his winning strategy; this would mean that Bob still has a
winning strategy from the resulting pair of states, that is $(x',y')
\in R$.
\end{itemize}
\end{itemize}

We have thus taken a concept which baffles postgraduate research
students, and presented it in a way which is well within the grasp of
first-year undergraduate students.

\paragraph{Determining who has the winning strategy}
\newblock

Once the notion of equivalence is understood in terms of winning
strategies in the Copy-Cat Game, the question then arises as to how to
determine if two particular states are equivalent, ie, if Bob has a
winning strategy starting with the tokens on the two given states.
This isn't generally a simple prospect; Games like Chess and Go are
notoriously difficult to play perfectly, as you can only look ahead a
few moves before getting caught up in the vast number of positions
into which the game may evolve.

Here again, though, we have a straightforward way to determine when
two states are equivalent.  Suppose we could paint the states of an
LTS in such a way that any two states which are equivalent -- that is,
from which Bob has a winning strategy -- are painted the same colour.

The following property would then hold.

\begin{quote}
\emph{If any state with some colour $C$
has a transition leading out of it
into a state with some colour $C'$,
then \emph{every} state with colour $C$
has an identically-labelled transition leading out of it
into a state coloured $C'$.}
\end{quote}

That is, if two tokens are on like-coloured states (meaning that Bob
has a winning strategy) then no matter what move Alice makes, Bob can
respond in such a way as to keep the tokens on like-coloured states
(ie, a position from which he still has a winning strategy).  We refer
to such a special colouring of the states a \emph{game colouring}.

To demonstrate, consider the following LTS.
\begin{center}
\setlength{\unitlength}{0.03mm}
\begin{picture}(1460,1500)(-30,50)
\linethickness{0.25pt}
\gameboard
\end{picture}
\end{center}

At the moment all states are coloured white, and we might consider
whether this is a valid game colouring.  It becomes readily apparent
that it is not, as the white state $4$ can make a $b$-transition to
the white state $5$ whereas none of the other white states ($1$, $2$,
$3$, $5$ and $6$) can do likewise.  In fact, in any game colouring,
the state $4$ must have a different colour from $1$, $2$, $3$, $5$ and
$6$.  Hence we paint it a different colour from white; in order to
present this example in black-and-white, we shall paint the state $4$
with the colour ``checkered.''

\begin{center}
\setlength{\unitlength}{0.03mm}
\begin{picture}(1460,1500)(-30,50)
\linethickness{0.25pt}
\gameboard
\put(780,490){\line(0,1){120}}
\put(760,470){\line(0,1){160}}
\put(740,458){\line(0,1){184}}
\put(720,452){\line(0,1){196}}
\put(700,450){\line(0,1){200}}
\put(680,452){\line(0,1){196}}
\put(660,458){\line(0,1){184}}
\put(640,470){\line(0,1){160}}
\put(620,490){\line(0,1){120}}

\put(640,630){\line(1,0){120}}
\put(620,610){\line(1,0){160}}
\put(608,590){\line(1,0){184}}
\put(602,570){\line(1,0){196}}
\put(600,550){\line(1,0){200}}
\put(602,530){\line(1,0){196}}
\put(608,510){\line(1,0){184}}
\put(620,490){\line(1,0){160}}
\put(640,470){\line(1,0){120}}

\end{picture}
\end{center}

We again consider whether this is now a valid game colouring.  Again
it becomes apparent that it is not, as the white states $3$ and $6$
have $a$-transitions to a checkered state, whereas none of the other
white states $1$, $2$ and $5$ do.  And in any game colouring, the
states $3$ and $6$ must have a different colour from $1$, $2$ and $5$.
Hence we paint these a different colour from white and checkered; we
shall choose the colour ``swirly.''

\begin{center}
\setlength{\unitlength}{0.03mm}
\begin{picture}(1460,1500)(-30,50)
\linethickness{0.25pt}
\gameboard
\put(780,490){\line(0,1){120}}
\put(760,470){\line(0,1){160}}
\put(740,458){\line(0,1){184}}
\put(720,452){\line(0,1){196}}
\put(700,450){\line(0,1){200}}
\put(680,452){\line(0,1){196}}
\put(660,458){\line(0,1){184}}
\put(640,470){\line(0,1){160}}
\put(620,490){\line(0,1){120}}

\put(640,630){\line(1,0){120}}
\put(620,610){\line(1,0){160}}
\put(608,590){\line(1,0){184}}
\put(602,570){\line(1,0){196}}
\put(600,550){\line(1,0){200}}
\put(602,530){\line(1,0){196}}
\put(608,510){\line(1,0){184}}
\put(620,490){\line(1,0){160}}
\put(640,470){\line(1,0){120}}

\put(700,1050){\circle{40}}
\put(700,1050){\circle{80}}
\put(700,1050){\circle{120}}
\put(700,1050){\circle{160}}
\put(1200,200){\circle{40}}
\put(1200,200){\circle{80}}
\put(1200,200){\circle{120}}
\put(1200,200){\circle{160}}
\end{picture}
\end{center}

We again consider whether this is now a valid game colouring.  This
time we find that it is, as every state can do exactly the same ting
as every other state of the same colour:

\begin{itemize}
\item
every white state has an $a$-transition to a white state
and an $a$-transition to a swirly state;
\item
every swirly state has an $a$-transition to a swirly state
and an $a$-transition to a checkered state;
\item
every checkered state has a $b$-transition to a white state.
\end{itemize}

At this point we have a complete understanding of the game, and can
say with certainty which states are equivalent to each other.  This is
an exercise which first-year students can happily carry out on
arbitrarily-complicated LTSs, which again gives testament to the
effectiveness of using games to great success in imparting difficult
theoretical concepts to first-year students -- in this case the
concept of partition refinement.

%While students take turns playing each other in the Copy-Cat game they
%develop an intuitive understanding of winning strategies: that the
%first player must play correctly, and the second player -- no matter
%how well they play -- can never win. They even have fun doing it! This
%allows them to argue when two systems are different (or the same) and
%even paves the way for other more advanced formal verification
%techniques such as observational equivalence.

\section{Conclusion}

Students can quickly and easily understand the modelling of computing
systems if it is done in a suitable way. Starting with some formal
semantics and real world examples, in our experience, makes the task
very daunting, difficult and generally unpleasant for
students. However, appealing to their existing understand of how the
world works, using puzzles as a medium, students can quickly become
comfortable using mathematical concepts such as LTSs. A similar lesson
is learnt when it comes to teaching verification: starting with the
formal definition of bisimulation (or similar) is an uphill battle
from the start, whereas starting from games like the Copy-Cat Game,
life is much easier for everyone.

We have used these approaches for over a decade to successfully teach
the modelling and verification of computing systems to first-year
students of our undergraduate course. This has eventually lead to the
production of our new modern textbook, to be published by Springer in
Autumn 2013, aimed at teaching first-year undergraduate students the
essential mathematics and modelling techniques for computing systems
in a novel and relatively light-weight way.


%\section*{Acknowledgements}


\bibliographystyle{ACM-Reference-Format}
\bibliography{ITiCSE2019} 

\end{document}
